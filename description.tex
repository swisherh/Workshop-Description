\documentclass[11pt]{article}

\usepackage{fullpage, amsmath, amssymb, url, color}

\newcommand{\C}{\mathbb C}
\newcommand{\Q}{\mathbb Q}
\newcommand{\Z}{\mathbb Z}
\newcommand{\R}{\mathbb R}

% for version control info
\newif\ifshowvc
\showvctrue
%\showvcfalse  % uncomment this line to turn off inclusion of vc info

\ifshowvc
\input{vc}
\fi


\begin{document}
\ifshowvc
\let\thefootnote\relax
\footnotetext{Base revision~\GITAbrHash, \GITAuthorDate,
\GITAuthorName.}
\fi


\begin{center}
\Large {\bf{\underline{Project Description}}} \\
%\medskip
%Computational Representation Theory in Number Theory Workshop\\
\end{center}

\medskip

\normalsize

\section{Introduction and Objectives}
This proposal is to request funding to hold a workshop at Oregon State University from July 27-31, 2015, titled {\em Computational Representation Theory in Number Theory}.

{\color{blue}
The theme for this workshop is \\

This workshop is particularly timely because....\\

A major objective of this workshop is to present recent work to a wider audience through the L-functions and Modular Forms Database (\textsf{LMFDB}).\\


\noindent Overall, the purpose of this workshop is threefold:
\begin{enumerate}
\item  To advance research in XXXXX, etc, etc
\item  To increase the participation of women, etc, etc...  XXXXX
\item To improve the LMFDB by adding data and information, expanding on existing data and information, and improving the user interface.
\end{enumerate}
}

\section{Scientific Program}

The focus for this workshop is...\\

We would like to have invited research talks about... by...\\


\begin{itemize}

\item
Tim Dokchitser Artin Representations

\item
Alex Bartel Intergral Representations

\item
Andrew Sutherland, (Massachusetts Institute of Technology)
\end{itemize}

\subsection{Artin representations}

Artin representations provide a good introduction to ideas about
$L$-functions and motives.  On one hand, they are relatively simple
and concrete.  On the other hand, there are still basic open problems
such as proving that their $L$-functions are always holomorphic (the
Artin conjecture) when the representation does not contain the trivial
representation.

On a computational level, there has been a fair amount of activity in
computing two dimensional icosahedral Galois representations and
matching them with weight one modular forms.  Often times, one will
work at the projective level of Galois representations to
$\textrm{PGL}_2(\C)$.  Recently Booker et.{}al.{} have been looking at
liftings of such representations to $\textrm{GL}_2(\C)$, and Jones and
Roberts have computed complete initial segments of lists of Artin
representations, i.e., for a given finite group $G$, faithful
irreducible complex representation $\rho$ of $G$, and bound $B$, they find all
Artin representations with the given Galois group/representation which
have conductor $\leq B$.

We plan on two to three lectures on this topic covering an
introduction to Artin representations and $L$-functions, the most
recent results on the Artin conjecture, and on computing Artin
representations.

There is preliminary work in the LMFDB on Artin representations.
During the afternoon sessions, we then hope to improve this area and
extend its data.

\subsection{Integral representations}

Integral representations, i.e., for a finite group $G$, we consider
homomorphisms $G\to \textrm{Aut}(\Z^r)$ for some $r>0$.  Since such
representations are never irreducible, we are primarily interested in
indecomposible representations.  Such representations occur naturally
in several ways.  For example, if $K/\Q$ is a finite Galois extension
with Galois group $G$, then $G$ acts on the ring of integers
$\mathcal{O}_K$, and on the torsion-free quotient of the unit group,
$\mathcal{O}_K^\times/\textrm{Tor}(\mathcal{O}_K^\times)$.  Moreover,
if $E$ is an elliptic curve over $\Q$, then $G$ acts on the
torsion-free quotient of the Mordell-Weil group
$E(K)/\textrm{Tors}(E(K))$.

Despite how well we understand the corresponding representations when
tensored with $\Q$, there has been recent work on classifying
indecomposible representations of small finite groups (such as $A_4$).
Moreover, Bartel and Lenstra have been studying arithmetic statistics
for unit groups modulo torsion.

{\color{magenta} A potential problem: I don't know if Bartel and Lenstra
  are ready for us to publicize their work.}

We envision one or two lectures on this topic covering an introduction
to integral representation theory, and then on more recent work.
There is some data in the LMFDB on the Galois module structure of unit
groups (modulo torsion).  During the afternoon sessions, we can look
to extend the range of this data.  Other groups may look at the
feasibility of adding Galois module data for the other two examples
mentioned above.

\subsection{Mod $p$ Galois representations}

Given an elliptic curve $E$ defined over $\Q$ and a prime $p$, the
absolute Galois group $G=\textrm{Gal}(\overline{\Q}/\Q)$ acts on the
group of $p$-torsion points $E[p]$.  Picking a basis for $E[p]$, we
get a representation $G\to \textrm{GL}_2(\Z/p\Z)$.  If the elliptic
curve does not have complex multiplication, then a classical result of
Serre shows that the representation is surjective for all but finitely
many primes.

Recently, Sutherland has computed the non-surjective representations
for many ellipitic curves in the LMFDB.  We propose one or two talks
in this area to describe the background, and then to explain how
computations are done.  In the afternoon sessions, groups may look at
extending the computations to mod $p$ Galois representations to
elliptic curves in the LMFDB which are defined over number fields
other than $\Q$.

\subsection{$\ell$-adic representations}

Algebraic varieties, and more generally motives, have families of
$\ell$-adic Galois representations associated to them.  Since the
definition of corresponding $L$-functions can be described in terms of
these representations, it would be natural to include them in the
LMFDB.  We propose having one introductory talk followed by work in
the afternoons on how to incorporate $\ell$-adic representations in
the LMFDB.


\subsection{Recent Meetings}
While there are numerous conferences and workshops in number theory, we are not aware of any recent conferences which have a significant overlap with this proposed workshop.  University of North Carolina at Greensboro held the {\it UNCG Summer School in Computational Number Theory} May 19-24, 2014.  However, this was a summer school geared toward students, rather than a professional workshop, and although a couple of the talks have overlap with our proposed focus, as a whole they lacked the level of concentration in the subject that our workshop will have.  In addition, by the time of our proposed workshop more than a year will have passed since the summer school.


\section{Organization}

The workshop will be held July 27-31, 2015 at Oregon State University in Corvallis, Oregon.  Corvallis is easily accessed by shuttle from either Portland International Airport (PDX) in Portland, OR, or Eugene Airport (EUG) in Eugene, OR.  

Oregon State University will provide meeting space and Wi-Fi access for all workshop activities, and the Department of Mathematics will provide office staff support for the workshop.


\subsection{Organizing Committee}
The members of the organizing committee are 
\begin{itemize}
\item Holly Swisher, committee chair (Oregon State University)
\item John Cremona (University of Warwick)
\item David Farmer (American Institute of Mathematics)
\item Paul Gunnells (University of Massachusetts, Amherst)
\item John Jones (Arizona State University)
\end{itemize}

\subsection{Logistics and Format}
The workshop will last five days, and will be a hybrid of typical conference and workshop formats.  The mornings will be dedicated to invited research talks, as well as discussions about the afternoon activities.  The afternoons will be dedicated to work in small groups.  Each afternoon some groups will work on research questions, some on generating computational data, and others on presenting computational work to the public.  A sample daily schedule is as follows.\\

\begin{tabular}{ll}
8:30 - 9:00 & Welcome and coffee \\
9:00 - 9:45 & Research Talk 1\\
9:45 - 10:00 & Break \\
10:00 - 10:45 & Research Talk 2\\
10:45 - 11:00 & Break \\
11:00 - 12:00 & Discussion, formation of groups for afternoon \\
12:00 - 1:30 & Lunch \\
1:30 - 3:30 & Work in Groups \\
3:30 - 4:00 & Break and Snacks\\
4:00 - 5:30  & Work in Groups \\
5:30 - 6:00 & Closing Discussion
\end{tabular}


\subsection{Dissemination of Results}
The results from this workshop will be disseminated through the $L$-functions and Modular Forms Database (\textsf{LMFDB}), located at \url{http://www.lmfdb.org/}.  This website is the result of an ongoing international collaboration with contributions from roughly $50$ number theorists thus far.  It serves the mathematical community in two ways.

First, it provides data that is the result of rigorous computations of mathematical objects of interest in the modern number theory community.  This data is organized around the theme of $L$-functions.  Objects of study include elliptic curves over the rationals, number fields, and various types of automorphic forms including classical, Hilbert, and Siegel.  Computational data of this sort has proven to be useful to researchers formulating and testing conjectures in number theory and in some cases can be used in the proofs of theorems.

Second, the \textsf{LMFDB} plays an educational role for people interested in number theory.  It includes an overview of number theoretic objects related to $L$-functions, contains brief explanations regarding the objects it contains, and shows interconnections between them.  For example, when viewing the web page for a specific elliptic curve over $\Q$, users see the important invariants of the curve such as its conductor, rank, and Mordell-Weil generators.  They also have links to pages for its associated modular form, isogenous curves, its $L$-function, and symmetric powers of its $L$-function.

During the afternoons, we will have participants extending and improving
the \textsf{LMFDB} in ways related to the themes of this workshop.  We expect to expand the \textsf{LMFDB} to include a section on $\ell$-adic Galois representations.  We also expect to improve and extend the areas on Artin representations, integral representations, mod $p$ Galois representations, and to improve the user interface aspects of the site.

\subsection{Publicity, Recruitment, and Support}
All funds requested are for participant support.  We aim to have close to 35 participants at the workshop.  This would be enough to ensure a rich research atmosphere, but still allow work in small groups to be manageable and productive.  We plan to invite {\color{blue}$28$} mathematicians we feel will be able to contribute in significant ways to the workshop, including invited speakers.  We are requesting funds to pay for travel, food, lodging, and registration for these participants.

\noindent Our initial list of invitees is given below.  It contains mathematicians at all stages of their careers from {\color{blue}graduate students to full professors, and one third of the group are women.}\\
{\color{blue}
\begin{tabular}{ll}
John Cremona & (University of Warwick), organizer \\
 David Farmer & (American Institute of Mathematics), organizer\\
 Paul Gunnells & (University of Massachusetts, Amherst), organizer\\
 John Jones & (Arizona State University), organizer\\
 *Jennifer Balakrishnan & (Oxford University),\\
 Alex Bartel & (University of Warwick) \\
 Jonathan Bober & (University of Bristol) \\
 *Alina Bucur & (UC San Diego) \\
 Brian Conrey & (American Institute of Mathematics) \\
 Paul-Olivier Dehaye & (University of Z\"urich) \\
 *Alyson Deines & (Center for Communications Research, La Jolla) \\
Tim Dokchitser & (University of Bristol) \\
 *Anna Haensch & (Duquesne University) \\
 *Sally Koutsoliotas & (Bucknell University) \\
 *Kristin Lauter & (Microsoft Research) \\
 Stefan Lemurell & (G\"oteborgs University) \\
 *Ling Long & (Louisiana State University) \\
*Melanie Matchett-Wood & (University of Wisconsin) \\
 Pascal Molin & (University of Paris 7) \\
 Cris Poor & (Fordham University) \\
 Michael Rubinstein & (Waterloo University) \\
 *Renate Scheidler & (University of Calgary) \\
 Harald Schilly & (University of Vienna) \\
 Andrew Sutherland & (Massachusetts Institute of Technology) \\
 William Stein & (University of Washington) \\
 Frederik Stromberg & (Durham University) \\
 Doug Ulmer & (Georgia Tech) \\
 *Bianca Viray & (University of Washington)\\
 *Christelle Vincent & (Stanford University)\\
 John Voight & (Dartmouth) \\
 Dan Yasaki & (UNC Greensboro) \\
 David Yuen & (Lake Forest) \\
\end{tabular}
}
\bigskip

We also request funding for the participation of $6$ non-invited participants.  Non-invited participants will be chosen from a pool of applicants with highest priority given to graduate students, then post-docs and junior faculty, and finally to established faculty who do not have external funding to support their travel.  Applicants from underrepresented groups will be considered one level higher in terms of priority.  We are requesting funds to pay for travel, shared lodging, and registration for these participants.

Attendance at the conference will be open to all mathematicians.  We
will advertise the conference through a variety of means:
\begin{itemize}
\item The list of conferences at \url{www.numbertheory.org}
\item The number theory listserv 
\item The Women in Numbers (WIN) listserv
\item A workshop website accessible to the public
\end{itemize}

We will solicit applications through the above means, and mention that we especially welcome applications from underrepresented groups.  In addition,  
the PI will use her connections with the Women in Sage network, in addition to the WIN network, to solicit applications from these groups.
  
\subsection{Child and Family Care}
In order to allow participation by mathematicians who have family care responsibilities, we have identified the following means for accessing local resources:
\begin{itemize}
\item {\bf Family Connections} is a Child Care Resource and Referral Agency based in nearby Albany, OR servicing the Corvallis/Albany area.  They offer personalized referrals for care, and problem solving.  
\item {\bf care.com/osu} is an online resource for finding child care, babysitting, and elder care; it also provides a search for local and other care providers.  As an OSU employee, the PI will have access to this resource and be able to share this information with workshop participants.
\item {\bf http://oregonstate.edu/childcare/} contains information about childcare and family resources at OSU, and includes a link to numerous off-campus centers and programs around the Corvallis area.
\end{itemize}  
This information will be made available on the conference website so that prospective and invited participants will have adequate information about child and family care options.


\section{Results from Current and Prior NSF support}

PI Holly Swisher has received prior conference funding from the NSF to support the 25th Annual Automorphic Forms Workshop held at Oregon State University March 23-26, 2011.  The funding was through grant DMS-1069292 {\it Automorphic Forms Workshop}, for \$14,700 for the period of February 1, 2011 to January 31, 2012.

\subsection{Intellectual Merit}
The annual Automorphic Forms Workshops have built a reputation as internationally recognized and respected conferences that are attended by leading experts in this broad area of mathematics.  Automorphic forms is a central subject in contemporary number theory with deep connections  to many areas across mathematics and the mathematical sciences including representation theory, combinatorics, and mathematical physics.

This workshop provided an opportunity for dissemination of results in automorphic forms, as well as a welcoming environment for students to become acquainted with this important and rapidly growing field.  It also provided an opportunity for networking and the creation of research collaborations in this field.  As is evidenced below in the schedule of talks, research across a wide area of number theory was represented, by many leading experts as well as students and junior researchers. The conference schedule was as follows:\\

\small
WEDNESDAY MARCH 23, 2011\\

\begin{tabular}{ll}
8:30-9:00 & Registration and Coffee\\
9:00 - 9:30 & {\bf Lloyd Kilford}, ``An overconvergent Jacquet-Langlands correspondence for weight 1\\ &  modular forms" \\
9:40 - 10:00 & {\bf Jeremy Rouse}, ``Explicit bounds for sums of squares"\\
10:10 - 10:40 & {\bf Dermot McCarthy}, ``Hypergeometric Functions and Fourier Coefficients of \\ & Modular Forms"\\
10:50 - 11:10 & {\bf Kate Thompson}, ``Algorithms for Computing Hilbert Symbols over Number Fields" \\
11:20 - 12:00 & {\bf Jyoti Sengupta}, ``Sum of Kloosterman sums over arithmetic progressions" \\

12:00 - 1:30 & Lunch\\

1:30 - 2:00 &{\bf Guillermo Mantilla-Soler}, ``Weight $1$ modular forms attached to totally real cubic \\ & number fields"\\
2:10 - 2:30 &{\bf John Lopez}, ``Two-divisibility of certain weakly holomorphic modular forms"\\
2:40 - 3:10 &{\bf Jitendra Bajpai}, ``Weakly Holomorphic Vector-Valued Modular Forms For \\ & Genus-zero Subgroups Of the Modular group"\\
3:20 - 4:00  &{\bf Dubi Kelmer}, ``A uniform spectral gap for congruence covers of a hyperbolic manifold"\\

4:00 - 4:30 & Coffee Break\\
4:30 - 5:30 & {\bf Panel: Navigating Career Transitions}\\
\end{tabular}
\bigskip

THURSDAY MARCH 24\\

\begin{tabular}{ll}
8:30-9:00 & Registration and Coffee\\
9:00 - 9:30 & {\bf Nils-Peter Skoruppa}, ``Jacobi forms of singular and critical weight on the full \\ & modular group" \\
9:40 - 10:00 & {\bf Karen Taylor}, ``Dirichlet Series and Automorphic Forms"\\
10:10 - 10:40 & {\bf Stephan Ehlen}, ``Twisted traces of CM values of modular functions"\\
10:50 - 11:10 & {\bf Stephanie Treneer}, ``Constructing Simultaneous Hecke Eigenforms"\\
11:20 - 12:00 & {\bf Abhishek Saha}, ``Vertical equidistribution for Siegel modular forms of degree 2"\\
12:00 - 1:30 & Lunch \\
1:30 - 2:00 & {\bf Matt Young}, ``Analytic studies of Ichino's formula on the $L^2$ norm of the pullback \\ & of a Saito-Kurokawa lift" \\
2:10 - 2:30 & {\bf Michael Griffin}, ``$U_p$ congruences modulo powers of primes"\\
2:40 - 3:10 & {\bf Jeff Breeding}, ``Local data for dimensions of Siegel cusp forms"\\
3:20 - 4:00 & {\bf Hatice Boylan}, ``Weil representations and Jacobi forms of singular weight over\\ &  number fields" \\
4:00 - 4:30 & Coffee Break\\
4:30 - 5:30 &  {\bf Panel: Research with Students: from attracting students to publishing}\\
\end{tabular}
\bigskip

FRIDAY MARCH 25\\

\begin{tabular}{ll}
8:30-9:00 & Registration and Coffee\\
9:00 - 9:40 & {\bf Zach Kent}, ``p-adic properties of the partition function"\\
9:50 - 10:10 & {\bf Robert Osburn}, ``Automorphic properties of combinatorial generating functions" \\
10:20 - 10:40 & {\bf Nick Andersen}, ``Divisibility Properties of Coefficients of Level p Modular\\ &  Functions for Genus Zero Primes"\\
10:40 - 11:00 & Coffee Break\\
11:00 - 11:20 & {\bf Haigang Zhou}, ``The Jacobi Eisenstein series of weight two" \\
11:30 - 11:50 & {\bf Benjamin Linowitz}, ``Decomposition theorems for Hilbert modular newforms"\\
12:00 - 12:20 & {\bf Jonathan Hanke}, ``S-genus identities for ternary quadratic theta series" \\
12:30 - 2:00 & Lunch\\
2:00 - & Free Afternoon\\
\end{tabular}
\bigskip

SATURDAY MARCH 26\\

\begin{tabular}{ll}
8:30-9:00 & Registration and Coffee\\
9:00 - 9:30 & {\bf Fredrik Str�mberg}, ``On non-existence of (truly) newforms on $\Gamma_0(9)$"\\
9:40 - 10:00 & {\bf Christelle Vincent}, ``Weierstrass points on the Drinfeld modular curve"\\
10:10 - 10:40 & {\bf Byungchul Cha}, ``Bounds of the summatory function of Moebius function in\\ &  function fields"\\
10:50 - 11:10 & {\bf Paul Jenkins}, ``Bounds for coefficients of cusp forms and extremal lattices"\\ 
12:00 - 1:30 & Lunch\\
1:30 - 2:00 & {\bf Kimball Martin}, ``Central L-values via the relative trace formula" \\
2:10 - 2:40 & {\bf Johnson Jia}, ``Fourier coefficients and Siegel theta series"\\
2:50 - 3:20 & {\bf Martin Hoevel}, ``Automorphic forms with singularities on the hyperbolic space"\\
3:30 - 3:50 & {\bf Michael Dewar}, ``The image and kernel of Atkin's $U_p$ operator mod p"\\
3:50 - 4:15 & Coffee Break\\
4:15 - 5:15 & {\bf Discussion: AFW 2012, 2013}
\end{tabular}
\bigskip

\normalsize

\subsection{Broader Impacts}

The Automorphic Forms Workshops have traditionally been incredibly welcome environments for junior researchers in automorphic forms and related areas, and this workshop was no exception.  Junior mathematicians, through giving talks to more experienced researchers, are able to get important and valuable feedback which promotes the training of future experts.  At the same time, the participation of international leading experts makes the scientific standards of the workshop particularly high. 
  
We had talks by undergraduate students, graduate students, post-docs, and professors of all levels.  Over half of the participants were students, post-docs, or pre-tenured researchers. 

In addition to conference talks, we also had two panel discussions to increase the professional development of the junior researchers at the workshop.  The first panel topic was Navigating Career Transitions.  The panelists were Tom Shemanske (Dartmouth), Stephanie Treneer (Western Washington University), and Matt Young (Texas A\&M).  Most career paths for mathematicians involve several important transitions. In this panel we discussed ways to successfully navigate:
\begin{itemize}
\item starting your first job
\item the tenure process
\item changing directions in research
\item changing academic jobs
\item transitioning between academics and industry
\end{itemize}

The second panel topic was Research with Students: from attracting students to publishing.  The panelists were Jim Brown (Clemson), Giuliana Davidoff (Mount Holyoke), Paul Jenkins (Brigham Young), and Holly Swisher (Oregon State).  In the discussion we addressed:
\begin{itemize}
\item attracting undergraduate, masters, and PhD students
\item choosing research problems for students
\item motivating and encouraging students throughout the research process
\item where to publish work with students
\end{itemize}
\medskip

Information about the conference including participants, schedule, panels, funding, and travel/lodging was made available, and is still available, at the following website. 
\begin{center}
\url{http://automorphicformsworkshop.org/pastworkshops/2011/index.html}
\end{center}
\end{document}
